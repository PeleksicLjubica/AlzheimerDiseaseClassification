\documentclass[12pt,oneside]{memoir} 
\usepackage[latinica]{matfmaster}
\usepackage[latinica]{pangrami}

% Datoteka sa literaturom u BibTex tj. BibLaTeX/Biber formatu
\bib{Master_rad}

\autor{Ljubica Peleksić}
\naslov{Klasifikacija obolelih od Alchajmerove bolesti na osnovu analize spontanog govora}
\godina{2021}

\mentor{doc. dr Jelena Graovac, profesor\\ Univerzitet u Beogradu, Matematički fakultet}

%TODO Dodati clanove komisije i datum odbrane
\komisijaA{prof. dr Gordana Pavlović-Lažetić}
\komisijaB{doc. dr Jovana Kovačević}
\datumodbrane{}

% Apstrakt na srpskom jeziku (u odabranom pismu)
\apstr{%
% TODO Dodati apstrakt
}


% Ključne reči na srpskom jeziku (u odabranom pismu)
\kljucnereci{klasifikacija, Alchajmer, NLP}

\begin{document}

% ==============================================================================
% Uvodni deo teze
\frontmatter
% ==============================================================================
% Naslovna strana
\naslovna
% Strana sa podacima o mentoru i članovima komisije
\komisija
% Strana sa posvetom (u odabranom pismu)
% TODO Dodati posvetu
\posveta{porodici}
% Strana sa podacima o disertaciji na srpskom jeziku
% Sadržaj teze
\tableofcontents*

% ==============================================================================
% Glavni deo teze
\mainmatter
% ==============================================================================

% ------------------------------------------------------------------------------

\chapter{Uvod}

U današnjem svetu covek ima kao i mogućnost da veliki broj procesa automatizuje korišćenjem računara.  Kako u drugim životnim oblastima,  ovo se ispoljava i u medicini.  Alchajmerova bolest je tip bolesti mozga koja sa vremenom postaje sve teža.  Simptomi se dešavaju kada su neuroni u delovima mozga koji su zaduženi za učenje i memoriju tj. kognitivne funkcije oštećeni. \cite{Alzheimer_facts_figures}. Smatra se da sama bolest krene 20 godina pre nego što se ispolje bilo kakvi simptomi.  Prvi problemi koji se pojavljuju kod obolelih su poteskoće u govoru, kao i gubitak memorije.  Kasnije u toku bolesti se pojavljuju i problemi pre obavljaju nekih jednostavih zadataka. 


\chapter{Podaci}

U ovom poglavlju će biti predstavljeni podaci koji su korišćeni u radu za dobijanje rezultata,  kao i tehnike obrade podataka koje su vršene nad njima.  Podaci korišćeni za rešavanje problema klasifikacije obolelih osoba od Alchajmerove bolesti su transkri-
pti slobodnog govora dementnih osoba. Transkripti se prikupljaju kao audio ili video zapisi,  a razgovori sa osobama nisu struktuirani niti imaju neki odredjeni sled.  Na osobi koja intervjuiše je da postavlja pitanja,  dok intervjuisani odgovara.  Osoba koja intervjuiše takodje prati tok razgovora sa pitanjima,  te nisu u svakom razgovoru ista.  
U svrhu poredjenja, pored podataka o osobama obolelim od demencije tipa Alchajmer,  bili su potrebni i intervjui sa nedementnim starijim osobama. Intervjui sa osobama obolelim od Alchajmera su sakupljani u obliku video i audio zapisa koji su prikupljeni od osoba koje su koristile usluge dnevnog boravka za obolele od Alchajmera u Novom Sadu,  jedine takve ustanove u Srbiji organizovane od strane Udruženja građana Alchajmer.  Ovi intervjui su bili prikpljani tokom grupnih razgovora sa obolelim.  Intervjui sa nedementnim starijim osobama su prikupljeni zahvaljujući više od 20 volontera studenata Biološkog fakulteta u Beogradu,  krajem 2017. godine i u toku 2018.
Od audio zapisa razgovora sa osobama obolelim od Alchajmera su kreirani transkripti od strane više osoba,  dok su studenti koji su učestvovali u prikupljanju podataka od dementnih starijih osoba zatim od audio zapisa kreirali tekstualne zapise.  Svaki intervju je zapisan na dva načina i to kao originalan u kome je svaka reč napisana tačno onako kako je izgovorena,  uključujući ponavljanja,  nedovršene reči,  greške u izgovoru i drugi u kome su ispravljene sve greške u izgovoru.  Kako su ovi intervjui sprovedeni grupno,  prvi korak je bio da se razdvoje razgovori,  tako da rečenice koje je izgovorila jedna osoba se nalaze u jednoj tekstualnoj datoteci koja nosi ime te osobe.
Reči osobe koja postavlja pitanja se ne moraju naći u svakoj datoteci i ne predstavljaju podatke koji se obradjuju.  Nakon toga transkripti su pažljivo obradjeni,  tako da svaki bude ispravno podeljen na rečenice.  Ovaj korak je bio krucijalan kako bi u narednom koraku moglo da se dodje do vrsta reči i lema,  za svaku izgovorenu reč.  

Transkripti su zapisivani po definisanom protokolu koji ubraja sledeće:

\begin{enumerate}
\item Intervju sa jednom osobom se nalazi u datoteci koja nosi ime te osobe
\item Ime obolelog se označava vitičastim zagradama 
\item Pitanje postavljeno od strane osobe koja vodi intervju se označava uglastim zagradama
\item Koristi se UFT-8 kodna šema i latinica
\item Koriste se slova sa dijakriticima (č, ć, š, đ,…)
\item Pauze između izgovorenih reči se zapisuju odgovarajućim brojem crtica,  gde svaka crtica predstavlja jedadan sekund pauze
\item Brojevi se zapisuju sa crticom izmedju,  ako su višecifreni brojevi
\end{enumerate}

Nakon što su transkripti bili ispravno zapisani po protokolu,  a rečenice ispravno podeljene,  takve datoteke su prosledjene na Filološki fakultet u Beogradu,  gde su odradjeni za svaku reč algoritamskim putem vrsta reči i njena lema.  Vrste reči biće upotrebljene u okviru rešavanja problema klasifikacije obolelih pacijenata od Alchajmerove bolesti leksičkim metodama.  Lema reči se koristi pri računanju vrednosti vokabulara i vokabulara za reči izgovorene jednom.  Ove metrike će biti pomenute u daljem radu, u sekciji koja se bavi rešavanjem problema metodama leksičke analize. 

\section{Problem odredjivanja vrsta reči}

U daljem tekstu će biti objašnjen problem dodeljivanja vrsta reči sekvenci reči,  kao i predstavljena dva algoritma za rešavanje ovog problema:
\begin{enumerate}
\item Skriveni Markovljev Model (eng. Hidden Markov Model / HMM)
\item Uslovljena nasumična polja (eng. Conditional Random Fields/CRF)
\end{enumerate}
\break
Vrste reči koje postoje i njigove oznake na engleskom jeziku su sledece:

\begin{enumerate}
\item Pridev: ADJ
\item Apozicija: ADP
\item Prilog: ADV
\item Pomoćni: AUX
\item Veznici: CCONJ
\item Član: DET
\item Uzvik: INTJ
\item Imenica: NOUN
\item Broj: NUM
\item Rečca: PART
\item Zamenica: PRON
\item Vlastita imenica: PROPN
\item Znak interpunkcije: PUNCT
\item Veznik: SCONJ
\item Simbol: SYM
\item Glagol: VERB
\item Drugo: X
\end{enumerate}

Metoda odredjivnja vrsta reči je kompleksan problem.  Zadatak je takav da dodeljujemo svakoj reči xi u ulaznoj sekvenci reči labelu yi,  tako da izlazna sekvenca Y ima istu dužinu kao ulazna sekvenca X.  \cite{pos_tagging}. Reči imaju vise mogućih vrsta reči i zadatak ovog procesa je da se pronadje ispravna vrsta reči za tu situaciju.  Za reči se bira takva vrsta koja je najverovatnija.  Za mnoge je verovatnoća da pripadaju svim vrstama sem jednoj izuzetno mala,  pa je lako odlučiti se.  

Jedan od načina da se reši problem odredjivanja vrsta reči je koristeći Skriveni Markovljev model (eng. Hidden Markov Model / HMM). Skriveni Markovljev model je probabilistički sekvencni model koji za sekvencu jedinica izračunava raspodelu verovatnoće po mogućim sekvencama i bira najbolju opciju.  Markovljev lanac,  model koji nam govori o verovatnoćama sekvenci slučajnih promenljivih,  ima pretpostavku da ako želimo da predvidimo buduće stanje,  jedino što je bitno je trenutno stanje.  Markovljev lanac se grafički predstavlja grafom, gde su čvorovi grafa stanja, a grane predstavljaju verovatnoće.  Suma svih vrednosti grana koje idu iz odredjenog čvora moraju biti 1.  Markovljev lanac se oslanja na dogadjaje koji mogu da se posmatraju,  dok u slučaju reči to nije moguće.  Zato se koriste Skriveni Markovljev Model., koji poseduje skrivene promenljive.  Zadatak odredjivanja skrivene sekvence promenljivih na osnovu observacija u modelu se naziva dekodiranje.  Skriveni Markovljev Model počiva na dve pretpostavke.  Prva je identična kao za Markovljev lanac, dok druga kaže da verovatnoća nekog stanja zavisi samo od stanja koje je proizvelo to stanje i ni jednog više.  Algoritam za odredjivanje vrsta reči se sastaji iz matrice koja sadrži verovatnoće da se jedna vrsta reči nalazi posle druge i matrice koja sadrži verovatnoće da se odredjena vrsta dodeli odredjenoj reči.  Algoritam za dekodiranje za Skriveni Markovljev Model se naziva Viterbi algoritam.  Viterbi algoritam prima dve matrice koje smo pomenuli, a vraća putanju kroz stanja Skrivenog Markovljevog Modela koja deeljuje najveću verovatnoću datoj sekvenci.  \cite{pos_tagging}.

Algoritam koji je prethodno prikazan ima problem sa nepoznatim recima,  vlastitim imenima,  akronimima,  novim rečima.  Algoritam uslovljenih nasumčnih polja nalazi način da iskoristi odredjene odlike reči,  npr. veliko slovo ili prefiks ili sufiks reči,  što je teško dodati u Skriveni Markovljev Model.  Trenira se logaritamski linearan model.  U modelu uslovljenih polja računamo verovatnoću svih vrsta reči u sekvenci,  a ne pojedinačni vrstu jedne po jedne reči.  Svako svojstvo se oslanja na vrstu reči prethodne i sledeće reči i na celu ulaznu sekvencu reči.  Za zaključivanje se takodje koristi Viterbi algoritam da bi se odabrala najbolja sekvenca vrsta reči.  \cite{pos_tagging}.

\section{Lematizacija}

Proces lematizacije jeste onaj u kome se za svaku reč nalazi njena kanonska forma, tj. lema. U srpskom jeziku,  za imenice lema je nominativ jednine,  za glagole infinitiv,  a za prideve nominativ jednine muškog roda. Takođe,  proces lematizacije uključuje vraćanje rodne varijacije reči na njen predjašnji oblik. 
Postoje dva metoda za rešavanje problema lematizacije.  Prvi da se prema obliku reči otklanja sufiks i nalazi njena lema.  Ovaj oblik rešavanja ne daje tako dobre rezultate.  Drugi pristup uključuje korišćenje skupa podataka koji za svaku reč,  za svaku njenu moguću vrstu reči,  ima odredjenu lemu.  Postoji mogućnost i kombinovanja ova dva pristupa. 

\section{Skup podataka pre i nekon obrade}

Skup podataka koji će biti korišćen za eksperimente u ovom radu sadrži sakupljene podatke na već opisan način,  koji su zatim obradjeni pomenutim tehinikama.  Intervjua sa pacijentima obolelim od demencije Alchajmerovog tipa ima 22,  koje nazivamo "pozitivni" i nalaze se u fascikli "P".  "Negativnih",  tj intervjua sa starijim licima koji nemaju utvrdjenu demenciju Alchajmerovog tipa ima 57 i oni se nalaze u fascikli "N".  Nakon procesa odredjivanja vrsta reči i lema za svaku reč intervjua,  za svaki transkript je kreiran novi dokument koji u sebi sadrzi potrebne informacije.  Ako se u transkriptu nalazi rečenica:
\newline

\noindent\fbox{%
    \parbox{\textwidth}{%
      Ja sam Bojana.  
    }%
}
\newline\newline
Onda će se u odgovarajućoj datoteci naći i sledeći redovi, gde u svakom prva reč označava izvorni oblik reči, druga vrstu, a treća lemu. 
\newline

\noindent\fbox{%
    \parbox{\textwidth}{%
      Ja	ADV	Ja\newline
	sam	AUX	jesam\newline
	Bojana	PROPN	Bojana\newline
	.	PUNCT	.
    }%
}

\chapter{Rešavanje problema metodama leksičke analize}

\chapter{Rešavanje problema metodama mašinskog učenja}

\section{Vektorska reprezentacija teksta}
\section{Vreća reči}
\section{N-grami}
\section{TF metrika}
\section{TF-IDF metrika}

\chapter{Rezultati}



\chapter{Zaključak}

\chapter{Test}

Ovo je rečenica u kojoj se javlja citat \cite{PetrovicMikic2015}.
Još jedan citat \cite{GuSh:243}.



% ------------------------------------------------------------------------------
% ------------------------------------------------------------------------------

% ------------------------------------------------------------------------------

% ------------------------------------------------------------------------------
% 
% ------------------------------------------------------------------------------
\literatura

% ==============================================================================
% Završni deo teze i prilozi
\backmatter
% ==============================================================================


% ------------------------------------------------------------------------------
% Biografija kandidata
\begin{biografija}
  \textbf{Ljubica Peleksić} (\emph{Beograd,
    18.  novembar 1993.}) 
	Ljubicina biografija
\end{biografija}
% ------------------------------------------------------------------------------


\end{document}